\documentclass[parskip,bibtotocnumbered]{scrreprt}

\usepackage{chngcntr}
\usepackage[multiple]{footmisc}

\usepackage{epigraph}

\deffootnote{1em}{1em}{\textsuperscript{\thefootnotemark}\ }
\counterwithout{footnote}{chapter}

\usepackage{setspace}

\usepackage{HSMW-Fonts}

\usepackage[all,blueheadings]{HSMW-Logo}
\usepackage{amsmath,amsthm}

\newtheorem{satz}{Satz}[chapter]
\renewcommand{\proofname}{Beweis}
\usepackage{url}

\usepackage[utf8]{inputenc}
\usepackage{verbatim}

\usepackage[colorlinks,linkcolor=HSMW-Blau,urlcolor=HSMW-Blau,anchorcolor=HSMW-Blau,citecolor=HSMW-Blau,breaklinks,bookmarksnumbered]{hyperref}

\hypersetup
{
pdftitle = {LaTeX-Vorlagen},
pdfsubject = {Hochschule Mittweida},
pdfauthor = {Klaus Dohmen},
pdfkeywords = {},
}

\author{Prof.\ K.\  Dohmen}
\publishers{\today}
\title{\Huge \scalebox{1.2}{\LaTeX-Vorlagen}}
\date{}

\addtocontents{toc}{\protect\onehalfspacing}

\usepackage{xspace}

\newcommand{\fn}[2]{\footnote{#2\label{fn:#1}}\xspace}
\newcommand*{\fnmark}[1]{\textsuperscript{\normalfont\ref{fn:#1}}\xspace}

\setcounter{tocdepth}{1}

\begin{document}
 
\pagenumbering{roman}

\maketitle

\chapter*{Vorwort}

\epigraph{\em I hope to die before I have to use Microsoft Word.}{Donald E. Knuth}

Zum Verst"andnis dieser Anleitung werden grundlegende \LaTeX{}-Kenntnisse
vorausgesetzt.  Das Lehrbuch von \textsc{Hedtke}, \textsc{Gippner} und
\textsc{M"uller} \cite{Hedtke:2009} bietet einen guten und preisg"unstigen
Einstieg in die \LaTeX{}-Welt.  Weitere Buchempfehlungen finden sich auf der
Webseite von \textsc{Dante} \cite{Dante:2009}, der Deutschsprachigen
Anwendervereinigung \TeX{} e.V.\, Empfehlenswert sind auch die Newsgroups
\texttt{comp.text.tex} (englisch) und \texttt{de.comp.text.tex} (deutsch), wo
man auf jede Frage nach kurzer Zeit eine oder mehrere qualifizierte Antworten
erh"alt.

Selbstverst"andlich nehme auch ich Ihre Fragen, Hinweise und Anregungen gerne
entgegen.  Ich w"unsche Ihnen viel Erfolg und Freude beim Erstellen Ihrer
Dokumente.

Happy \TeX{}ing!

\newcommand*\backupvskip{}
\let\backupvskip\chapterheadstartvskip
\renewcommand*\chapterheadstartvskip{\vspace{-\baselineskip}}
\tableofcontents
\let\chapterheadstartvskip\backupvskip


\chapter{Graduierungsarbeiten mit HSMW-Thesis}
\label{grad}

\pagenumbering{arabic}

Die Dokumentenklasse \texttt{HSMW-Thesis} dient der Erstellung von
Graduierungsarbeiten.  Sie erf"ullt detailgenau die Vorgaben des Corporate
Designs \cite{cdmanual}.

\section{Verwendung}
\label{verwendung}

Die Klasse \verb!HSMW-Thesis! kann sowohl mit \LaTeX{}, gefolgt von
\texttt{dvips}, als auch mit pdf\LaTeX{} genutzt werden.

Die Felder in der Pr"aambel sind entsprechend ihrer Bedeutung auszuf"ullen.
Sie k"onnen auch leer bleiben, was der Entfernung der entsprechenden Anweisung
gleichkommt.

Es wird standardm"a"sig das \texttt{babel}-Paket mit der Option
\texttt{ngerman} geladen.
\begin{verbatim}
\documentclass{HSMW-Thesis}
\Art{Bachelorarbeit}
\Thema{Beweis der Riemannschen Vermutung}
\Anrede{Herr} \Vorname{} \Nachname{Musterstudent}
\Fakultaet{Mathematik/Naturwissenschaften/Informatik}
\Studiengang{Angewandte Mathematik}
\Seminargruppe{MA09w1}
\Erstpruefer{Prof\@. Dr\@. Peter Tittmann}
\Zweitpruefer{Prof\@. Dr\@. Klaus Dohmen}
\Tag{1} \Monat{Juni} \Jahr{2015}
\Anlagen{}
\Copyright{Dieses Werk ist urheberrechtlich gesch"utzt.}
\Textsatz{} 
\Druck{} 
\Verlag{} 
\ISBN{}

\begin{document}

\begin{Referat}
% Referat
\end{Referat}

\begin{Vorwort}
% Vorwort
\end{Vorwort}

\Hauptteil
% Diese Anweisung nicht entfernen!

\chapter{Einleitung}

\Anhang

\begin{thebibliography}{99}
% Literaturangaben
\end{thebibliography}

\end{document}
\end{verbatim}
Per \verb! pdflatex -shell-escape <Dateiname> ! erh"alt man nun aus der
Quelldatei ein druckfertiges PDF-Dokument.  Die Option
\,\verb!-shell-escape!\, kann in der jeweils verwendeten \LaTeX{}-Umgebung
voreingestellt werden.  Diese Option und das Programm \texttt{makeindex}
werden ben"otigt, um das Abk"urzungs- und das Stichwortverzeichnis sowie
gegebenenfalls ein Glossar automatisch zu generieren.  Hierzu ist es
erforderlich, dass der Dokumentenklasse die Optionen {\ttfamily
  no"-men"-cla"-ture}, \texttt{index} bzw\@. \texttt{glossary} "ubergeben
werden.

Abk"urzungen k"onnen mit
\verb!\nomenclature{!\emph{Abk"urzung\/}\verb!}{!\emph{Bedeutung\/}\verb!}!
und Stichw"orter mit \verb!\index{!\emph{Stichwort\/}\verb!}!  in das
Abk"urzungs- bzw\@. Stichwortverzeichnis aufgenommen werden.  Zur korrekten
Wiedergabe der Seitenzahlen sind in der Regel zwei \LaTeX{}-L"aufe
erforderlich.  Weitere Informationen k"onnen den Dokumentationen der Pakete
{\ttfamily no"-mencl} (Abk"urzungsverzeichnis) und \texttt{makeidx}
(Stichwortverzeichnis) entnommen werden.  Man beachte, dass diese Pakete bei
Angabe der entsprechenden Optionen aus Abschnitt \ref{Optionen} automatisch
geladen werden, ebenso wie das Paket \texttt{glossaries}.  Eine Eintragung in
das Glossar erfolgt durch
\verb!\newglossaryentry{!\emph{Label\/}\verb!}{name=!\emph{Begriff}\verb!,!
  \verb!description={!\emph{Beschreibung}\verb!}}!  und die anschlie"sende
Ver"-wendung des Labels, z.B\@. per \verb!\gls{!\emph{Label\/}\verb!}!.

Mit der Anweisung \verb!\Firmenlogo{\put(x-coord,y-coord){...}}!  in der
Pr"aambel des Dokumentes kann ein zus"atzliches Firmenlogo auf der Titelseite
untergebracht werden.

\section{Optionen}
\label{Optionen}

Die Dokumentenklasse wird mit
\verb!\documentclass[!\emph{Optionen}\verb!]{HSMW-Thesis}! geladen, wobei
folgende Optionen angegeben werden k"onnen:

\begin{labeling}{\texttt{tablecaptionabove~\,}}

\item[\texttt{draft}] Automatische Zeilennummerierung und Anzeige der Labels
in Referenzen.  Es werden nur der eigentliche Text (ohne Vorwort) und das
Literaturverzeichnis gesetzt.

\item[\texttt{english}] F"ur englischsprachige Graduierungsarbeiten.  Das
\texttt{babel}-Paket wird mit der Option \texttt{english} geladen.

\item[\texttt{hypertext}] Erzeugung eines PDF-Dokumentes mit Hyperlinks

\item[\texttt{oneside}] Einseitiges Layout

\item[\texttt{noromanprefix}] "Uberschriften im Vorspann ohne vorangestellte
r"omische Ziffer

\item[\texttt{smallromans}] Paginierung im Vorspann mit kleinen r"omischen
Ziffern.  Die Paginierung beginnt mit dem Inhaltsverzeichnis auf Seite v.
Diese Option ist zu empfehlen, falls sich eines der Verzeichnisse im Vorspann
oder das Vorwort "uber mehr als eine Seite erstreckt.

\item[] Mit \texttt{smallromans=1} beginnt das Inhaltsverzeichnis auf Seite i.
Auch andere Anfangszahlen sind m"oglich.

\item[\texttt{classictoc}] Hierarchisches Layout des Inhaltsverzeichnisses

\item[\texttt{noseplists}] Abbildungs- und Tabellenverzeichnis auf derselben
Seite

\item[\texttt{nofigures}] Kein Abbildungsverzeichnis

\item[\texttt{notables}] Kein Tabellenverzeichnis

\item[\texttt{tablesfirst}] Tabellenverzeichnis vor dem Abbildungsverzeichnis

\item[\texttt{nomenclature}] Abk"urzungsverzeichnis

\item[\texttt{norefpage}] Keine Seitenzahlen im Abk"urzungsverzeichnis

\item[\texttt{counttables}] In den bibliografischen Angaben wird zus"atzlich
die Anzahl der Tabellen ausgewiesen.

\item[\texttt{minus}] Minuszeichen als Trennzeichen in Gleitumgebungen

\item[\texttt{tablecaptionabove}] Beschriftung oberhalb der Abbildung
bzw\@. Tabelle

\item[\texttt{openbib}] Offenes Layout f"ur das Literaturverzeichnis

\item[\texttt{index}] Stichwortverzeichnis

\item[\texttt{glossary}] Glossar

\item[\texttt{bigheadings}] Gr"o"sere Kapitel"uberschriften

\item[\texttt{smallheadings}] Kleinere Kapitel"uberschriften (nur bei
Pr"asentationen)

\item[\texttt{texnote}] Hinweis in den bibliografischen Angaben, dass das
Dokument mit \LaTeX{} erzeugt wurde

\item[\texttt{sansmath}] Mathematik-Symbole in serifenloser Schrift
 
\item[\texttt{thmbelow}] Zeilenvorschub und vertikaler Zwischenraum nach
Theorem-""\"ahn"-lichen Umgebungen

\item[\texttt{thmcolon}] Nummerierungen von Theorem-"ahnlichen Umgebungen
enden mit einem Doppelpunkt.

\item[\texttt{thmpoint}] Nummerierungen von Theorem-"ahnlichen Umgebungen
enden mit einem Punkt.

\item[\texttt{leqno}] Linksseitige Nummerierung mathematischer Gleichungen

\item[\texttt{fleqn}] Linksseitige Ausrichtung mathematischer Gleichungen

\item[\texttt{Eid}] Eidesstattliche Versicherung

\item[\texttt{150}] Jubil"aumslogo
\end{labeling}

Die Pakete \verb!amsmath!, \verb!amssymb! und \verb!amsthm! werden automatisch
geladen.  Folgende Theorem-"ahnliche Umgebungen werden durch die
Dokumentenklasse \verb!HSMW-Thesis! vordefiniert:
\begin{center}
\begin{tabular}{llll}
\texttt{theorem} & \texttt{hilfssatz} & \texttt{satz} & \texttt{folgerung} \\
\texttt{corollary} & \texttt{lemma} & \texttt{festlegung} & \texttt{definition} \\
\texttt{beispiel} & \texttt{example} & \texttt{bemerkung} & \texttt{remark}
\end{tabular}
\end{center}
Hinzu kommen die Sternvarianten \verb!theorem*!, \verb!satz*!, usw., bei denen
die fortlaufende Nummerierung unterdr"uckt wird.  Au"ser\-dem stehen f"ur
Beweise die Umgebungen \verb!beweis! und \verb!proof!, die beide ein
optionales Argument akzeptieren, zur Verf"ugung.

\chapter{Pr"asentationen mit HSMW-Beamer}
\label{HSMW-Beamer}

Mit \texttt{HSMW-Beamer} k"onnen Pr"asentationen unter Verwendung des
\LaTeX/Beamer-Paketes entsprechend den CD-Vorgaben erstellt werden.

\section{Verwendung}

Die Verwendung von \verb!HSMW-Beamer!  ist nur mit pdf\LaTeX{} m"oglich.
Allgemein wird die Nutzung des Beamer-Paketes in \cite{Tantau:2007} und
\cite{Voss:2009b} beschrieben.

Das folgende Beispiel zeigt die Verwendung von \verb!HSMW-Beamer!.  Die
Stilvorlage wird mit \verb!\usetheme{Mittweida}! im Vorspann eines
\LaTeX/Beamer- Dokumentes aktiviert.  Standardm"a"sig wird das Paket
\texttt{babel} mit den Optionen \texttt{english} und \texttt{ngerman} geladen.
\begin{verbatim}
\documentclass{beamer}

\usetheme{Mittweida}

\author{Name des Vortragenden}
\title{Titel des Vortrags}
\subtitle{Untertitel}
\institute{Instituts- oder Fakult"atsbezeichnung}
\date{Datum}

\begin{document}

\maketitle

\begin{frame}{Erste Folie}
...
\end{frame}

...
\end{document}
\end{verbatim}
Die Anweisungen \verb!\author!, \verb!\title! und \verb!\institute!
akzeptieren je ein optionales Argument, mit dem eine Kurzform der
entsprechenden Eintr"age festgelegt werden kann.

Eine absolute Positionierung ist mit der \verb!textblock!-Umgebung m"oglich:
\begin{verbatim}
\begin{textblock}{Breite}(x-Koordinate,y-Koordinate)
...
\end{textblock}
\end{verbatim}
Die Breite der Box sowie deren $x$- und $y$-Koordinate werden prozentual
angegeben.  Die Koordinaten beziehen sich auf die obere linke Ecke der Box.
Koordinatenursprungspunkt ist die obere linke Ecke des Bildschirms.
Beispielsweise erzeugt
\begin{verbatim}
\begin{textblock}{20}(50,50)
...
\end{textblock}
\end{verbatim}
eine Box, die 20\,\% der horizontalen Breite in Anspruch nimmt und deren obere
linke Ecke sich in der Mitte des Bildschirms befindet.

Eine relative Positionierung kann mit
\verb!\put(x-Differenz,y-Differenz){...}! erfolgen.

\section{Optionen}

Der \verb!HSMW-Beamer!-Stilvorlage kann in der \verb!\usetheme!-Anweisung eine
Liste von Optionen "ubergeben werden.  Folgende Optionen stehen zur
Verf"ugung:

\medskip
\begin{labeling}{\texttt{titlegraphic=Bilddatei}~\,}
\item[\texttt{Fakultaet\/$n$}] Layout entsprechend der Fakult"at $n$
($n=1,\dots,5$)

\item[\texttt{CB}] \"aquivalent zur Option \texttt{Fakultaet2}

\item[\texttt{plain}] schlichtes Layout entsprechend CD-Vorgabe

\item[\texttt{titlegraphic=none}] Kein Hintergrundbild auf der Titelseite
(Abk.: \texttt{notitlegraphic})

\item[\texttt{titlegraphic=Bilddatei}] \texttt{Bilddatei} als Hintergrundbild
auf der Titelseite

\item[\texttt{titlepage=none}] Keine Titelseite (Abk.: \texttt{notitlepage})

\item[\texttt{titlepage=Bilddatei}] \texttt{Bilddatei} als Titelseite mit
vorhandenen CD-Elementen

\item[\texttt{slogan=\{Text\}}] Festlegung des Slogans auf der Titelseite.

\item[\texttt{english}] F"ur englischsprachige Pr"asentationen.  Die
\verb!babel!-Sprachoption \verb!english! wird aktiviert.

\item[\texttt{colthm}] Farbige Definitionen, S\"atze, Beweise usw.

\item[\texttt{colmath}] Farbige Formeln

\item[\texttt{contents}] Inhaltsverzeichnis im HSMW-Panel\footnote{Es wird
immer die Langform der Abschnitts"uberschriften angezeigt.};
\verb!\nextframenocontents! unterdr"uckt das Inhaltsverzeichnis auf der
n"achsten Folie

\item[\texttt{navigation}] Aktivierung der Navigationsbuttons im HSMW-Panel

\item[\texttt{totalpages}] Angabe der Gesamtzahl der Seiten

\item[\texttt{prefix=\{Text\}}] Pr"afix f"ur die Paginierung

\item[\texttt{delimiter=\{Text\}}] Trennsymbol zwischen Paginierung und
Gesamtseitenzahl

\item[\texttt{progressbar}] Fortschrittsbalken f"ur Frames

\item[\texttt{progressbar=slide}] Fortschrittsbalken f"ur Overlays
\end{labeling}
S"amtliche Optionen wirken global.  F"ur die Optionen \verb!colthm!,
\verb!colmath! und \verb!progressbar! gibt es lokale Varianten in Form einer
gleichnamigen Umgebung.

Der Umgebung \verb!progressbar! kann optional das Argument \verb!slide!
"ubergeben werden.  Alternativ kann der Befehl \verb!\nextframeprogressbar!
verwendet werden, der ebenfalls "uber das optionale Argument \verb!slide!
verf"ugt.

Durch Angabe der Option \verb!handout! in
\verb!\documentclass[handout]{beamer}!  wird die Erstellung einer speziellen
Handout-Version veranlasst.  In der Voreinstellung der {\ttfamily
HSMW-""Bea"-mer}""-Stilvorlage werden zwei Folien auf eine DIN-A4-Seite
gedruckt.  Dabei wird ein schlichtes, druckerfreundliches Layout entsprechend
der Option \verb!plain! verwendet.

\section{Pr"asentationen mit zwei Beamern}

Die \verb!HSMW-Beamer!-Klasse unterst"utzt die Nutzung von zwei unabh"angigen
Bea\-mern, z.B\@. f"ur bilinguale Pr"asentationen oder Pr"asentationen, in
denen die aktuelle und die vorherige Folie gleichzeitig gezeigt werden.  Die
Projektionsbilder der beiden Beamer werden dabei nebeneinander auf dem
Notebook des Dozenten dargestellt, was u.a\@. auch f"ur nicht sichtbare
Notizen genutzt werden kann.  Mehr dazu in einem eigenen Artikel
\cite{Dohmen:2010b}.

An der HSMW ist der H"orsaal 2-205 mit zwei fest installierten Beamern
ausgestattet.

\section{Textform}

\vspace{-\bigskipamount} Mit Hilfe des Paketes \verb!beamerarticle! ist es
m"oglich, aus einer Pr"asentation, die mit der LaTeX/Beamer-Klasse erstellt
wurde, ein herk"ommliches \LaTeX-Textdokument zu erzeugen.  Die Vorgehensweise
wird in \cite{Tantau:2007} und \cite{Voss:2009b} ausf"uhrlich beschrieben.
"Ublicherweise kommen hierbei die \LaTeX{}-Standardklassen und die
KOMA-Script-Klassen zum Einsatz.

Die \verb!HSMW-Thesis!-Klasse wurde so konzipiert, dass sie mit
\verb!beamerarticle! gemeinsam genutzt werden kann.  Dazu ist in der Pr"aambel
des Dokumentes die \verb!beamer!-Klasse gegen die \verb!HSMW-Thesis!-Klasse
auszutauschen und zus"atzlich das Paket \verb!beamerarticle! zu laden.
Gegebenenfalls sind Eintragungen in den Feldern der \verb!HSMW-Thesis!-Klasse
vorzunehmen.  Auf diese Weise wird automatisiert eine schriftliche
Ausarbeitung generiert: \vspace{-.5\baselineskip}
\begin{verbatim}
\documentclass{HSMW-Thesis} 

\usepackage{beamerarticle}

\Art{Seminarvortrag}
\Thema{Genetische Algorithmen}
...

\begin{document}

\maketitle

\begin{frame}{Einf"uhrung}
...
\end{frame}

...
\end{document}
\end{verbatim}


\chapter{Pr"asentationen mit HSMW-QuickTalk}
\label{HSMW-QuickTalk}

Mit \verb!HSMW-QuickTalk!  kann aus einem gew"ohnlichen Text"-do"-ku"-ment
automatisiert eine PDF-""Pr"a"-sen"-ta"-tion in Anlehnung an die Vorgaben des
Corporate Designs erstellt werden.

\section{Verwendung}
\label{HSMWQuickieGrad}

Die Verwendung von \verb!HSMW-QuickTalk! ist nur mit pdf\LaTeX{} m"oglich.
Das Paket ist kompatibel mit den Standardklassen und den KOMA-Script-Klassen.
Es muss lediglich die Anweisung \verb!\usepackage{HSMW-QuickTalk}! zur
Pr"aambel hinzugef"ugt werden.  Dabei ist da\-rauf zu achten, dass
\texttt{HSMW-QuickTalk} vor allen anderen Paketen geladen wird.

Das Einf"ugen zus"atzlicher Folien ist mit der Umgebung \verb!slide! m"oglich.
Die "Uberschrift kann als optionales Argument "ubergeben werden.

\section{Optionen}

Das Paket \texttt{HSMW-QuickTalk} akzeptiert folgende Optionen:

\begin{labeling}{\texttt{delimiter=\{Text\}}\,~}
\item[\texttt{Autor=\{Text\}}] Autor der Pr"asentation

\item[\texttt{Titel=\{Text\}}] Titel der Pr"asentation

\item[\texttt{Datum=\{Text\}}] Datum der Pr"asentation

\item[\texttt{Fakultaet\/$n$}] Layout entsprechend der Fakult"at $n$
($n=1,\dots,5$)

\item[\texttt{CB}] \"aquivalent zur Option \texttt{Fakultaet2}

\item[\texttt{colmath}] Farbige Formeln im Text und in AMS\TeX-Umgebungen

\item[\texttt{footnotes}] Fu"snoten werden angezeigt.

\item[\texttt{notitlepage}] Keine Titelseite

\item[\texttt{abstract}] Anzeige der Zusammenfassung

\item[\texttt{nonavigation}] Deaktivierung der Navigationsbuttons

\item[\texttt{totalpages}] Angabe der Gesamtzahl der Seiten

\item[\texttt{prefix=\{Text\}}] Pr"afix f"ur die Paginierung

\item[\texttt{delimiter=\{Text\}}] Trennsymbol zwischen Paginierung und
Gesamtzahl der Seiten

\item[\texttt{nosans}] Keine automatische Umschaltung auf serifenlose Schrift
\end{labeling}

\chapter{Konferenzposter mit HSMW-Poster}

Mit \verb!HSMW-Poster! k"onnen wissenschaftliche Poster im
Querformat\footnote{Ich w"urde mich sehr dar"uber freuen, wenn jemand der
  Klasse \texttt{\footnotesize HSMW-Poster} eine Option f"ur eine Darstellung
  im Hochformat hinzuf"ugen w"urde.} in Anlehnung an das Corporate Design
erstellt werden.

\section{Verwendung}

Die Verwendung der Dokumentenklasse \verb!HSMW-Poster! ist nur mit pdf\LaTeX{}
m"oglich.  Die Einbindung des Paketes \verb!HSMW-Fonts! (siehe Abschnitt
\ref{hsmwfonts} auf Seite \pageref{hsmwfonts}) wird empfohlen.
Standardm"a"sig wird das Paket \verb!babel! mit den Optionen \verb!english!
und \verb!ngerman!  geladen. Es stehen s"amtliche Anweisungen aus dem
Beamer-Paket zur Verf"ugung.  Die Umgebung \verb!poster! ersetzt die
\verb!frame!-Umgebung und wird im Regelfall nur einmal verwendet.
\begin{verbatim}
\documentclass[A1]{HSMW-Poster}

% \usepackage{HSMW-Fonts} 

\author{Klaus Dohmen}
\title{Improved Bonferroni Inequalities via Graphs and Abstract Tubes} 
\conference{European Conference on Combinatorics, Graph Theory %
  and Applications}
\date{Oct 5--9, 2005}

\begin{document}

\begin{poster}
...
\end{poster}

\end{document}
\end{verbatim}

\section{Optionen}

Die Dokumentenklasse \texttt{HSMW-Poster} akzeptiert folgende Optionen:

\begin{labeling}{\texttt{Fakultaet $3$}~\,}

\item[\texttt{A\/$n$}] DIN-Gr"o"se des Posters ($n=0,\dots,4$)

\item[\texttt{Fakultaet\/$n$}] Layout entsprechend der Fakult"at $n$
($n=1,\dots,5$)

\item[\texttt{CB}] \"aquivalent zur Option \texttt{Fakultaet2}

\item[\texttt{colthm}] Farbige Definitionen, S\"atze, Beweise usw.

\item[\texttt{colmath}] Farbige Formeln

\end{labeling}

\section{Drucken}

Hat man keinen Drucker zur Verf"ugung, der bis zum Rand druckt, so muss die
Ausgabedatei vor dem Drucken skaliert werden.  Hierzu kann beispielsweise das
Linux-\-Komman\-do\-zei\-len\-tool \verb!pdfnup! verwendet werden.  Bei
A4-Folien erh"alt man mit
\begin{verbatim}
pdfnup --nup 1x1 --trim "-0.5cm -0.5cm -0.5cm -0.5cm" \ 
  --outfile out.pdf source.pdf
\end{verbatim}
gute Resultate. Die Ma"se sind entsprechend dem jeweiligen Drucker anzupassen.

\appendix

\addtocontents{toc}{\protect\setcounter{tocdepth}{0}}
\addtocontents{toc}{\protect\medskip}
\addtocontents{toc}{\protect\contentsline{chapter}{Anhang}{}{}}

\chapter{Weitere Klassen und Pakete}

\section{HSMW-Brief}
\label{briefinst}

Die Klasse \texttt{HSMW-Brief} dient der Erstellung von Dienstbriefen.  Einen
eigenen Briefkopf erh"alt man durch Umbenennung und "Anderung der Datei
\verb!Dohmen.lco!.  

Die Klasse \texttt{HSMW-Brief} l"adt das Paket \texttt{babel} mit den Optionen
\texttt{english} und \texttt{ngerman}.  Sie nutzt, wenn m"oglich, die
Hausschrift UniversLT (siehe Abschnitt \ref{hsmwfonts}), sonst Helvetica.  Das
nachfolgende Beispiel zeigt die Verwendung der Klasse \verb!HSMW-Brief!:
\begin{verbatim}
\documentclass[Dohmen]{HSMW-Brief}

\datum{}

\begin{document}

\begin{brief}{Empf"angeradresse}
\betreff{Betreffzeile}
\anrede{Sehr geehrte ...}
...
\gruss*{Mit freundlichen Gr"u"sen}{}
%\anlage{} oder \anlagen{}
%\verteiler{}
\end{brief}

\end{document}
\end{verbatim}
Die Sternvariante der \verb!\gruss!-Anweisung f"ugt die eigene Unterschrift als Bild (\texttt{png} oder \texttt{eps}) ein.
F"ur englischsprachige Briefe stehen \verb!\encl{}! und \verb!\cc{}! zur Verf"ugung.

Bis auf die selbsterkl"arenden Optionen \texttt{extralogos},
\texttt{noextralogos}, \texttt{moreextralogos}, \texttt{150}, \texttt{C4},
\texttt{DL}, \texttt{qrcode} und \texttt{noqrcode} werden s"amtliche
Klassenoptionen an die nachgeladene KOMA-Script-Klasse {\ttfamily scr"-lettr2}
weitergereicht.  N"aheres hierzu findet man in der Anleitung zu KOMA-Script
\cite{Kohm:2008}.  Die Hochschulfarbe steht als \texttt{HSMW-Blau} zur
Verf"ugung.

\section{HSMW-Worksheet}
\label{worksheetinst}

Die Dokumentenklasse \texttt{HSMW-Worksheet} dient der Erstellung von
Aufgabenbl"attern mit L"osungen.  Die Ausgabe der L"osungen kann durch eine
Option gesteuert werden.

Empfehlenswert ist die zus"atzliche Installation des Paketes
\texttt{HSMW-Fonts} (siehe Abschnitt \ref{hsmwfonts} auf Seite
\pageref{hsmwfonts}).  Das nachfolgende Beispiel zeigt die Nutzung der Klasse
\texttt{HSMW-Worksheet}:
\begin{verbatim}
\documentclass[solutions]{HSMW-Worksheet}

\Fakultaet{Fakult\"at MNI}
\Dozent{Prof\@. K. Dohmen}
\Lehreinheit{\"Ubungen zur Diskreten Mathematik}
\Semester[WS 2012/13]{Wintersemester 2012/13}
\Nummer{Blatt 11}
\Thema{Graphen und Netzwerke}
\Aufgaben{8} % nur fuer Klausuren

\begin{document}

\begin{aufgabe}[\stern\ (2 Punkte)]
Zeigen Sie:
\begin{teilaufgaben}
\teilaufgabe Jeder endliche Baum hat mindestens einen Knoten vom Grad 1.
\teilaufgabe Jeder Baum mit $n$ Knoten besitzt $n-1$ Kanten.
\end{teilaufgaben}
\end{aufgabe}

\loesungskasten[L"osung zu a):]{4cm} % nur fuer Klausuren

\begin{loesung}
Mit vollst"andiger Induktion nach der Anzahl der Knoten.
\end{loesung}

\begin{aufgabe}[~~(8 Punkte) \Zusatzblatt]
Wahr oder falsch? Begr"unden Sie Ihre Antwort!
\begin{wahrfalsch}[i)]
  \wf{Es gibt einen zusammenh"angenden Graphen mit 21 Knoten und 19 Kanten.}
  \wf{Jeder 4-f"arbbare Graph ist planar.}
  \wf{Jeder planare Graph besitzt die chromatische Zahl~4.}
\end{wahrfalsch}
\end{aufgabe}

\end{document}
\end{verbatim}
Die Dokumentenklasse \texttt{HSMW-Worksheet} verf"ugt "uber folgende Optionen:
\begin{labeling}{\texttt{solutions=multicols}~\,}
 \item[\texttt{solutions}] separate Seite mit L"osungen
 \item[\texttt{solutions=columns}] separate, zweispaltige Seite mit L"osungen
 \item[\texttt{solutions=insitu}] L"osungen in situ
 \item[\texttt{hints}] separate Seite mit L"osungshinweisen
 \item[\texttt{hints=columns}] separate, zweispaltige Seite mit
L"osungshinweisen
 \item[\texttt{hints=insitu}] L"osungshinweise in situ
 \item[\texttt{seplines}] Ausgabe von mit \verb!\sepline! erzeugten
Trennlinien
 \item[\texttt{sansserif}] Aufgabentexte und Formeln in serifenloser Schrift
 \item[\texttt{hypertext}] Erzeugung eines PDF-Dokumentes mit Hyperlinks
 \item[\texttt{draft}] Unterdr"uckung des Hochschullogos
 \item[\texttt{english}] f"ur englischsprachige "Ubungsbl"atter (keine
Klausuren)
\end{labeling} Weitere Optionen werden an die Dokumentenklasse \verb!scrreprt!
weitergereicht.  Zur Strukturierung des Aufgabenblattes stehen folgende
Umgebungen zur Verf"ugung:
\begin{center}
\begin{tabular}{llll} \texttt{aufgabe} & \texttt{teilaufgaben} &
\texttt{loesung} & \texttt{loesungshinweis} \\ \texttt{exercise} &
\texttt{subexercises} & \texttt{solution} & \texttt{hint} \\
\end{tabular}
\end{center} Hinzu kommen die Sternvarianten \verb!aufgabe*!, \verb!loesung*!,
usw., bei denen eine evtl\@. vorhandene Nummerierung unterdr"uckt wird.  Die
Umgebungen \texttt{teilaufgaben} und {\ttfamily sub"-exer"-ci"-ses}
akzeptieren ein optionales Argument entsprechend dem \texttt{enumerate}-Paket.
Innerhalb dieser Umgebungen stehen die Gliederungsbefehle \verb!\teilaufgabe!
und \verb!\subexercise! zur Verf"ugung, die ebenfalls ein optionales Argument
entsprechend dem \verb!enumerate!-Paket (vgl\@. \verb!\item!) akzeptieren.
Mit der Option \verb!skip! k"onnen Gliederungsnummern in den Teilaufgaben
"ubersprungen werden.  Geht der ersten Teilaufgabe kein Text voraus, so sorgt
\verb!\phantomtext! f"ur den richtigen Zeilenabstand.  Durch ein optionales
Argument kann dieser als Vielfaches von \verb!\baselineskip! explizit
angegeben werden (Vorgabe: 1.63).

F"ur bedingtes Kompilieren stehen die \verb!\if!-Anweisungen
\verb!\ifsolutions!, \verb!\ifnosolutions!, \verb!\ifhints!,
\verb!\ifnohints!, \verb!\ifdraft!, \verb!\ifhypertext!, \verb!\ifgerman! und
\verb!\ifenglish! zur Verf"ugung.  Der Befehl
\verb!\translate{!\textit{deutscher Text}~\,\verb!>>!~\textit{englischer
    Text}\verb!}! f"ugt je nachdem, ob die Option \verb!english! gesetzt ist
oder nicht, den englischen bzw\@. den deutschen Text ein.  Mit Hilfe der
\texttt{comment}-Um"-ge"-bung k"onnen ganze Textbl"ocke auskommentiert werden.

Die Mathematik-Pakete \texttt{amsmath}, \texttt{amssymb} und \texttt{amsthm}
der American Mathematical Society werden von der Dokumentenklasse {\ttfamily
  HSMW-""Work"-sheet} standardm"a"sig hinzugeladen.  Unter Verwendung des
\texttt{multicol}-Paketes, welches ebenfalls standardm"a"sig hinzugeladen
wird, k"onnen Aufgaben mehrspaltig gesetzt werden.

\section{HSMW-Faltblatt}

Mit der Dokumentenklasse \texttt{HSMW-Faltblatt} k"onnen Faltbl"atter im
DIN-A3- und DIN-A4-Format erstellt werden.  Diese bestehen aus sechs logischen
Seiten, die auf die Vorder- und R"uckseite eines Blattes gedruckt werden.  Auf
Seite~1 erscheint das Hochschullogo. Weitere Grafiken k"onnen mit
\verb!\AddToBackground{!\emph{Seitennummer}\verb!}{!\emph{\LaTeX-Picture-Anweisungen}\verb!}!
an beliebiger Stelle positioniert werden.  Optionen werden an die
\texttt{article}- und \texttt{leaflet}-Klasse weitergereicht.  Folgende
weitere Optionen werden akzeptiert:
\begin{labeling}{\texttt{blueheadings}~\,}
 \item[\texttt{blueheadings}] Blaue Abschnitts"uberschriften\fn{2}{Die Farbe
der "Uberschriften wird als \texttt{\small HSMW-Blau} verf"ugbar gemacht.}
 \item[\texttt{nofoldmark}] Keine Faltmarken
 \item[\texttt{notumble}] R"uckseite um 180\textdegree{} gedreht
 \item[150] Jubil"aumslogo
\end{labeling}

\section{HSMW-Logo}

Das Paket \texttt{HSMW-Logo} dient der Erstellung von Dokumenten, die auf der
Titelseite und evtl\@. allen nachfolgenden Seiten das Hochschullogo tragen.
Das folgende Minimalbeispiel zeigt die Nutzung des Paketes bei Verwendung der
KOMA-Script-Klasse \verb!scrartcl!:

\begin{verbatim}
\documentclass{scrartcl}
\usepackage{HSMW-Logo}
\begin{document}
...
\end{document}
\end{verbatim}
Das Paket \texttt{HSMW-Logo} akzeptiert folgende Optionen:
\begin{labeling}{\texttt{blueheadings}~\,}
 \item[\texttt{all}] Logo auf allen Seiten (auf den Folgeseiten etwas kleiner)
 \item[\texttt{blueheadings}] Blaue Gliederungs"uberschriften bei Nutzung der
KOMA-Script-Klassen\footnote{Die Farbe der "Uberschriften wird als
\texttt{\small HSMW-Blau} verf"ugbar gemacht.}
 \item[150] Jubil"aumslogo 
\end{labeling}

\section{HSMW-Fonts}
\label{hsmwfonts}

Die HSMW-Fonts\footnote{Bezugsquelle: siehe \cite{cdmanual}.}  k"onnen mittels
\texttt{otftotfm} \cite{Kohler:2010} installiert werden:
\vspace{-.5\baselineskip}
\begin{verbatim}
        otftotfm -a -e texnansx UniversLTPro-55Roman.otf \
                 -fkern -fliga LY1--UniversLTPro-55Roman
\end{verbatim}
\vspace{-.5\baselineskip} Entsprechend f"ur die anderen Schriftschnitte der
HSMW-Fonts.  Hierzu sind in der Regel Administratorrechte erforderlich.
Au"serdem wird das Archiv \texttt{HSMW-Fonts.tar.gz} ben"otigt, welches im
\TeX-Ver"-zeich"-nis"-baum entpackt wird.  Nach einer Aktualisierung des
\TeX-Ver"-zeich"-nis"-bau"-mes stehen die Fonts per
\verb!\usepackage{HSMW-Fonts}! zur Verf"ugung.

Bei Verwendung der KOMA-Script-Klassen werden entsprechend den Festlegungen
des Corporate Designs "Uberschriften in Futura und der Flie"stext in Univers
gesetzt.  Un"-ab"-h"ang"-ig von der verwendeten Klasse kann mit
\verb!\sffamily! zu Futura und mit \verb!\rmfamily! zu Univers gewechselt
werden.  Entsprechendes gilt f"ur \verb!\textsf! und \verb!\textrm!.

Das Paket \texttt{HSMW-Fonts} akzeptiert folgende Optionen:

\begin{labeling}{\texttt{blueheadings}~\,}
 \item[\texttt{blueheadings}] Blaue Gliederungs"uberschriften bei Nutzung der
KOMA-Script-Klassen und der \texttt{HSMW-Faltblatt}-Klasse\fnmark{2}
 \item[\texttt{futuralight}] Verwendung der Schriftschnitte \emph{Light} und
\emph{Book} statt \emph{Book} und \emph{Bold}
\end{labeling}

\enlargethispage*{1.1\baselineskip}

\textbf{Hinweis:}~\,Die HSMW-Fonts eignen sich nicht zum Setzen mathematischer
Texte.  Sie sollten daher nicht in Verbindung mit der
\texttt{HSMW-Thesis}-Klasse verwendet werden.

\renewcommand{\bibname}{Literatur- und Quellenverzeichnis}
\begin{thebibliography}{11}
\enlargethispage{\baselineskip}
\label{LitVZ}

\bibitem{Braune:2006} \textsc{K. Braune, J. Lammarsch, M. Lammarsch}:
\emph{\LaTeX{} -- Basissystem, Layout, Formelsatz}, Springer-Verlag, Berlin,
Heidelberg, 2006.

\bibitem{Dante:2009} \textsc{Dante e.V.,} Deutschsprachige Anwendervereinigung
\TeX{} e.V., URL: \url{http://www.dante.de}, Stand: \today.

\bibitem{Dohmen:2010b} \textsc{K. Dohmen}: \emph{Dual Screen Presentations
  with the \LaTeX{} Beamer Class under X}, Special Issue on \LaTeX{} Academic
Workbench, The Prac\TeX{} Journal, 2010, No.~1, 7 S\@., URL:
\url{http://www.tug.org/pracjourn/2010-1/dohmen/dohmen.pdf}.

\bibitem{Hedtke:2009} \textsc{I. Hedtke}, \textsc{D. Gippner},
\textsc{R. M"uller}: \emph{Der \LaTeX-Tutor}, Shaker Media, Aachen, 2009.

\bibitem{cdmanual} \textsc{Hochschule Mittweida, AG Internet:} \emph{Das neue
  CD}, URL: \url{http://www.agi.hs-mittweida.de}, Stand: \today.

\bibitem{Kohler:2010} \textsc{E. Kohler:} \emph{otftotfm Manual}, LCDF Type
Software, URL: \url{http://www.lcdf.org/type}, Stand: \today.

\bibitem{Kohm:2009} \textsc{M. Kohm}: \emph{KOMA-Script}, CTAN:
\href{http://mirror.ctan.org/macros/latex/contrib/koma-script}{\texttt{macros/latex/contrib/koma-script}},
Stand: \today.

\bibitem{Kohm:2008} \textsc{M. Kohm}, \textsc{J.-U. Morawski}:
\emph{KOMA-Script -- Die Anleitung}, 3. erw\@. Auflage, Lehmanns Media,
Berlin, 2008.

\bibitem{leaflet} \textsc{R. Niepraschk, W. Schmidt, H. G"a"slein:} \emph{The
  leaflet document class}, CTAN:
\href{http://mirror.ctan.org/macros/latex/contrib/leaflet}{\texttt{macros/latex/contrib/leaflet}},
Stand: \today.

\bibitem{Tantau:2007} \textsc{T. Tantau}: \emph{The beamer class}, CTAN:
\href{http://mirror.ctan.org/macros/latex/contrib/beamer}{\texttt{macros/latex/contrib/beamer}},
Stand: \today.

\bibitem{Voss:2009b} \textsc{H. Voss}: \emph{Pr"asentationen mit \LaTeX},
Lehmanns Media, Berlin, 2009.

\bibitem{Voss:2009c} \textsc{H. Voss}: \emph{Mathematiksatz mit \LaTeX},
Lehmanns Media, Berlin, 2009.

\end{thebibliography}

\end{document}

%%% Local Variables:
%%% mode: latex
%%% TeX-master: t
%%% End:
